%\setcounter{secnumdepth}{-1}
    \chapter{Dot product}
    \section{Introduction}
    The dot product or scalar product is an algebraic operation that takes two equal length sequence of numbers and return a single number.\\
 The dot product of two vectors a = [a1, a2, ..., an] and b = [b1, b2, ..., bn] is defined as:\\
 \begin{equation}
 	\mathbf {a} \cdot \mathbf {b} =\sum_{i=1}^{n}a_{i}b_{i}=a_{1}b_{1}+a_{2}b_{2}+\cdots +a_{n}b_{n}
 \end{equation}
%$\mathbf {a} \cdot \mathbf {b} =\sum_{i=1}^{n}a_{i}b_{i}=a_{1}b_{1}+a_{2}b_{2}+\cdots +a_{n}b_{n}$\\
where $\sum$ denotes summation and n is the dimension of the vector space.\\
\begin{equation}
	\begin{bmatrix}1&3&-5\end{bmatrix}
    	\begin{bmatrix}4\\-2\\-1\end{bmatrix}= 1*4+3*(-2)+(-5)*(-1) = 3.
\end{equation}
%    	$\begin{bmatrix}1&3&-5\end{bmatrix}
%    	\begin{bmatrix}4\\-2\\-1\end{bmatrix}= 1*4+3*(-2)+(-5)*(-1) = 3.$\\
	For dot product, we have to same length of both vectors.
    \section{CPU implementation}
    The CPU implementation is below:\\
    \begin{lstlisting}[language=C, caption=vector dot product in C++]
    	for i < size of vector;i++
    		result += vector1[i] * vector2[i]
    \end{lstlisting}
    As we can see, It is simple to implement in CPU. We walk through the vector1 and vector2 and  multiply the same position elements of both vectors and add result in previous result.
 	\section{OCCA implementation / Reduction}
 	Kernel implementation of dot product is below.
 	\begin{lstlisting}[language=C, caption=vector dot product in OCCA]
 for (int group = 0; group < ((entries + block - 1) / block); ++group; @outer) {
        shared float s_vec[256];
        for (int item = 0; item < block; ++item; @inner) {
            if ((group * block + item) < entries) {
                s_vec[item] = vec[group * block + item] * vec2[group * block + item];
            } else {
                s_vec[item] = 0;
            }
        }
		for (int alive = ((block + 1) / 2); 0 < alive; alive /= 2) {
            for (int item = 0; item < block; ++item; @inner) {
                if (item < alive) {
                    s_vec[item] += s_vec[item + alive];
                }
            }
            for (int item = 0; item < block; ++item; @inner) {
                if (item == 0) {
                    blockSum[group] = s_vec[0];
                }
            }
        }
    }
 	\end{lstlisting}
 	We implement partial reduction of of vector using loop tiles of size block. In this case block is 256. As we can see firstly we save elements multiplication in shared memory. Now it will produce a vector of length block. Now in next loop it alive point to middle of the vector and item point 0 of vector. we add the elements of the vector and save it on position 0. In next loop, we just copy result to blocksum. 
 	\section{GPU vs CPU}
 	As we can estimate GPU will more expensive than CPU. Because in GPU firstly we have to copy the multiplication to the  shared memory. After that we start the partial reduction . And we know that memory transfer in GPU is more expensive than CPU. Another reason is that we implemented with partial reduction in GPU, but in CPU we are implement without reduction. 
 	
