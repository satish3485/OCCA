\setcounter{secnumdepth}{-1}
    \chapter{Conclusion}

This thesis, we develop some algorithms with OCCA and C++ by comparing the performances. The time performance of the implementations shows that OCCA with GPU is normally faster than C++ and CPU.  In all cases, the timing results recommend the usage of OCCA instead of the CPU to get  a better performance when the input size of the matrices is large. In our case, We got better performance in all algorithms. But with sparse matrices the performance are not always good. OCCA with GPU is slower than C++ and CPU; because at the moment it not support atomic operation, and other advanced parallel programming features. In some methods, due to OCCA limitation, we can not do parallel computation and we have to return to CPU and call back to kernel that is time consuming.\\
OCCA is easy to use and understand. OCCA uses a syntax similar to C. It has similar for loops (with additional field) and conditional statements. It is good solution, If somebody want to develop parallel applications for different architectures like openCL and CUDA (it works with same code for both architectures). We do not need to write separate code for OpenCL and Cuda. We have shared or exclusive memory operations in OCCA.

The problem of OCCA is that at the moment, the main structure of the project is not completely defined. The biggest problem is the lack of documentation. It has just 2 or 3 slide of documentation and 7 or 8, very short examples. So It is hard to understand, how to implement an optimal kernel.\\
For example in this project, I need atomic operations. But at the moment are not yet supported by OCCA. And the syntax of the kernel is not yet completely defined. For example during the development of this project, the OCCA developers has decided to change a little the syntax of the kernel. And I have spent some time  to  adapt all the methods.

Currently, OCCA developers are working on loop-carried dependency analysis for kernel generation + testing and support offset in kernel calls. They are also developing tiling loop labels for example tile(x,y) and possibly loop-collapsing. And working also on atomic operations. \\

I think OCCA is good solution for parallel programming with GPU. Because an application can works on all different devices like Nvidia, Intel and Radeon GPUs. So the developers are not constrained to a specific family of products. But at the moment, as described, the big problem of OCCA is that is a work progress. So, It is very interesting language, but now, it is not recommendable for long term or big projects.   